%%% Yiguang Lu CV-ENG Template
%%%
%%% Date: March 2013

%%%%%%%%%%%%%%%%%%%%%%%%%%%%%%%%%%%%%
% Document properties and packages
%%%%%%%%%%%%%%%%%%%%%%%%%%%%%%%%%%%%%
\documentclass[a4paper,12pt,final]{memoir}

% misc
\renewcommand{\familydefault}{bch}	% font
\pagestyle{empty}					% no pagenumbering
\setlength{\parindent}{0pt}			% no paragraph indentation
\setlength{\parskip}{0pt}
\linespread{1.4}

% required packages
\usepackage[hmargin=1.25cm, vmargin=2.0cm]{geometry}   % margins
\usepackage{url}									 % URLs
\usepackage{hyperref}
\usepackage[usenames,dvipsnames]{xcolor}				 % color
\usepackage{fontspec,xltxtra,xunicode}
\usepackage{multicol}								 % columns env.
  \setlength{\multicolsep}{0pt}
\usepackage{paralist}								 % compact lists
\usepackage{engord}

% for Chinese
\XeTeXlinebreaklocale "zh"
%%%%%%%%%%%%%%%%%%%%%%%%%%%%%%%%%%%%%
% Commands from fontspec
%%%%%%%%%%%%%%%%%%%%%%%%%%%%%%%%%%%%%
\defaultfontfeatures{Mapping=tex-text}
\newfontfamily \brandonfont{Heiti SC Light}
\newfontfamily \oslenfont{Microsoft YaHei}
\newfontfamily \gillfont{Microsoft YaHei}

\newcommand \redcolor{\color{Cerulean}}
\newcommand \graycolor{\color{Gray}}
\newcommand \blackcolor{\color{Black}}

\newcommand \brandonred{\brandonfont\redcolor}
\newcommand \brandongray{\brandonfont\graycolor}
\newcommand \brandonblack{\brandonfont\blackcolor}

\newcommand \oslenred{\oslenfont\redcolor}
\newcommand \oslengray{\oslenfont\graycolor}
\newcommand \oslenblack{\oslenfont\blackcolor}

\newcommand \gillred{\gillfont\redcolor}
\newcommand \gillgray{\gillfont\graycolor}
\newcommand \gillblack{\gillfont\blackcolor}
%%%%%%%%%%%%%%%%%%%%%%%%%%%%%%%%%%%%%
% Define macros (for convience)
%%%%%%%%%%%%%%%%%%%%%%%%%%%%%%%%%%%%%
\newcommand{\CVContactKey}[1]
{\small\brandonred\textbf{#1}}

\newcommand{\CVContactValue}[1]
	{\small\brandonblack{#1}}

\newcommand{\CVSection}[1]
	{\small\brandonred\textbf{#1}\\[6pt]}

\newcommand{\CVSkill}[1]
	{\scriptsize\brandonblack{>>\quad #1}\\[8pt]}

\newcommand{\CVAward}[1]
	{\CVSkill{#1}}

\newcommand{\CVExam}[1]
	{\CVSkill{#1}}

\newcommand{\CVRightTime}[1]
	{\small\oslenblack\textbf{#1}}

\newcommand{\CVRightTitle}[1]
	{\CVRightTime{#1}}

\newcommand{\CVRightSubtitle}[1]
	{\footnotesize\oslengray{#1}}

\newcommand{\CVRightMain}[1]
	{\scriptsize\oslenblack{#1}}

\newcommand{\CVEducationSection}[1]
	{\footnotesize\oslenblack\textbf{#1}}

\newcommand{\CVExperienceIntro}[1]
	{\CVRightMain{\textbf{#1}}\\[8pt]}

\newcommand{\CVExperienceAdvisor}[1]
	{\CVExperienceIntro{Advisor: #1}}

\newcommand{\CVExperienceItem}[1]
	{\CVRightMain{{\redcolor$\circ$}\quad {#1}}\\[4pt]}

\newcommand{\CVRightNumber}[1]
	{\footnotesize\brandonblack\textbf{#1}}

\newcommand{\CVUpdateDate}[1]
	{\scriptsize\brandonblack{#1}\\}
%%%%%%%%%%%%%%%%%%%%%%%%%%%%%%%%%%%%%
% Begin document
%%%%%%%%%%%%%%%%%%%%%%%%%%%%%%%%%%%%%
\begin{document}
% First column
%%%%%%%%%%%%%%%%%%%%%%%%%%%%%%%%%%%%%
\begin{minipage}[t]{0.62\textwidth}
  \begin{flushleft}
    \vspace{0pt}
    \gillfont\Huge{鲁倚光}\\[-4pt]
    \brandongray\normalsize{软件开发}
    \\[21pt]
  \end{flushleft}
  % Education
  %%%%%%%%%%%%%%%%%%%%%%%%%%%%%%%%%%%%%
  \CVSection{教育程度}
  \begin{tabular}{@{}ll@{}}
    \CVRightTime{2009--现今} & \CVRightTitle{同济大学软件学院,软件工程专业}\\[-4pt]
                                & \CVRightSubtitle{软件工程学士}\\
  \end{tabular}
  \\[10pt]
  \CVEducationSection{在校学习课程}\\[-10pt]
  \CVRightMain{}
  \begin{multicols}{3}
    \begin{compactitem}[\color{Cerulean}$\circ$]
    \item 离散数学
    \item 线性代数

    \item C语言程序设计
    \item 面向对象程序设计

    \item 算法与数据结构
    \item 算法分析与设计

    \item 操作系统
    \item Linux程序设计

    \item 计算机网络
    \item 计算机组成原理

    \item 数据库原理与设计
    \item 软件工程

    \item 软件架构与设计模式
    \item 计算机体系结构
    \end{compactitem}
  \end{multicols}
  \mbox{}\\[13pt]
  % Experience
  %%%%%%%%%%%%%%%%%%%%%%%%%%%%%%%%%%%%%
  \CVSection{学习、实习、就业经历}
  % Internship, BMW
  %%%%%%%%%%%%%%%%%%%%%%%%%%%%%%%%%%%%%
  \begin{tabular}{@{}ll@{}}
    \CVRightTime{2012.4--2013.5} & \CVRightTitle{宝马中国,上海互联驾驶实验室}\\[-6pt]
                                 & \CVRightSubtitle{软件开发实习生}\\
  \end{tabular}
  \\[10pt]
  \CVExperienceIntro{在宝马互联驾驶实验室实习的过程中,我得到了很多团队协作相关的经验,也学到了很多解决问题的方法。}
  \CVExperienceItem{曾参与{\redcolor{GEVINI}}在Android上的实现Demo的开发。} 
  \CVExperienceItem{给{\redcolor{KaFAS}}以策略优化、bug修复,并为其做一些性能测试。}
    \CVExperienceItem{在{\redcolor{KaFAS}}路测的过程中收集、整理并分析测试数据。}
  \\[13pt]
  % Tilt
  %%%%%%%%%%%%%%%%%%%%%%%%%%%%%%%%%%%%%
  \begin{tabular}{@{}ll@{}}
    \CVRightTime{2011.11--2012.1} & \CVRightTitle{\href{http://github.com/lukmy/tilt}{Tilt}, 同济大学,课程项目}\\
                                  & \CVRightSubtitle{项目组长,主要贡献者}\\
  \end{tabular}
  \\[10pt]
  \CVExperienceIntro{Tilt是个基于Git思想设计的版本控制系统,可以用来管理项目源码,控制项目进度。}
  \CVExperienceItem{主要负责软件架构的设计和基本源码的实现。}
  \CVExperienceItem{开始使用C来实现,后来改用C++和Boost对源码进行重写。}
  \CVExperienceItem{此项目是{\redcolor{软件工程课程}}和{\redcolor{Linux程序设计课程}}的课程项目。} \\[13pt]
  % Tilt
  %%%%%%%%%%%%%%%%%%%%%%%%%%%%%%%%%%%%%
  \begin{tabular}{@{}ll@{}}
    \CVRightTime{2011.6--2011.11} & \CVRightTitle{3D CubeCube,同济大学,谷歌俱乐部}\\
                                  & \CVRightSubtitle{主要贡献者}\\
  \end{tabular}
  \\[10pt]
  \CVExperienceIntro{3D CubeCube是一个有挑战性的三维四子棋游戏。}
  \CVExperienceItem{此程序参加了2011年谷歌大学生Android应用挑战赛,获得上海赛区预赛三等奖。}
  \CVExperienceItem{完成应用设计AI部分代码,使用了位运算加速解决方案的推导。}
\end{minipage}
\hfill
% Second column
%%%%%%%%%%%%%%%%%%%%%%%%%%%%%%%%%%%%%
\begin{minipage}[t]{0.32\textwidth}
  \vspace{51pt}
  \begin{tabular}{@{}ll@{}}
    \CVContactKey{联系电话} & \CVContactValue{(86)186-9216-5516}\\[-1pt]
    \CVContactKey{电子邮件} & \CVContactValue{\href{mailto:lukmylyg@gmail.com}{lukmylyg@gmail.com}}\\[-1pt]
    \CVContactKey{GitHub} & \CVContactValue{\href{http://github.com/lukmy}{github.com/lukmy}}\\[-1pt]
  \end{tabular} 
  \\[33pt]
  % Skills
  %%%%%%%%%%%%%%%%%%%%%%%%%%%%%%%%%%%%%
  \CVSection{掌握技巧}
  \CVSkill{超过两年的{\redcolor{Linux}}使用和管理经验,使用熟悉过{\redcolor{Ubuntu}}、{\redcolor{Arch}}、{\redcolor{Gentoo}}等发行版。}
  \CVSkill{大约三年的{\redcolor{Vim}}使用经历,重度Vim使用者,知道一些{\redcolor{Vim Script}},同时也使用{\redcolor{Sublime Text}}。}
  \CVSkill{比较好的掌握一些{\redcolor{Git}}的使用技巧,曾经尝试过从头实现Git。}
  \CVSkill{比较了解一些诸如{\redcolor{C}}、{\redcolor{Java}}、{\redcolor{C++}}、{\redcolor{Pascal}}的编译语言。能够比较好的使用{\redcolor{STL}}和{\redcolor{Boost}}完成一些工作。}
  \CVSkill{对于诸如{\redcolor{Ruby}}、{\redcolor{Perl}}、{\redcolor{Python}}、{\redcolor{Shell Script}}的脚本语言有一些基本的认识。}
  \CVSkill{粗浅的知道一些{\redcolor{\LaTeX{}}}。}
  \CVSkill{有一些{\redcolor{Android}}开发的经验。}\\[17pt]
  % Awards
  %%%%%%%%%%%%%%%%%%%%%%%%%%%%%%%%%%%%%
  \CVSection{获得奖项}
  \CVAward{(2012.7) 2012年英特尔杯全国大学生嵌入式挑战赛,决赛三等奖。}
  \CVAward{(2011.9) 2011年谷歌大学生Android应用开发挑战赛,上海赛区预赛三等奖。}
  \CVAward{(2007.9) 全国青少年信息学奥林匹克信息大赛,湖南赛区一等奖。}
\end{minipage} 
%%%%%%%%%%%%%%%%%%%%%%%%%%%%%%%%%%%%%
% End document
%%%%%%%%%%%%%%%%%%%%%%%%%%%%%%%%%%%%%
\end{document}
